\chapter{Einleitung und Kontext}

Im Rahmen des Praktikums Softwaretechnik II haben wir eine Spielleiteranwendung für das Rollenspiel \glqq Die Werwölfe von Düsterwald\grqq{} programmiert. 

\medskip
Bei diesem Spiel sitzen die Mitspieler im Kreis. Sie verkörpern in jeder Spielrunde einen Charakter, den sie zu Beginn jeder Spielrunde verteilt bekommen. Dieser ist den anderen Spielern nicht bekannt. Außerdem können sie einen Beruf ausüben, dieser ist, anders als der Charakter, auch den Mitspielern bekannt. \\
Die Charaktere und Berufe dürfen während der Partie verschiedenste Aktionen durchführen. 

\medskip
Die Charaktere haben unterschiedliche Ziele. Charaktere mit gleichem Ziel gehören einer Fraktion an. Ziel einer Fraktion ist es, ihr Gewinnziel zu erreichen. Dann hat die entsprechende Fraktion gewonnen und die Spielrunde endet. 

\medskip
Die Zielgruppe für die Software sind keine Anfänger, die das Spiel noch erlernen wollen. \\
Der Spielleiter sollte bereits einen ersten Überblick über die Regeln haben, bestenfalls bereits eine handvoll Spielrunden geleitet haben. 

\medskip
Da sich der Spielleiter hierbei unter Umständen sehr viel merken muss und dies meist nur mit einer weiteren Person gemeinsam bewerkstelligen kann, soll die Software ihn durch die Partie leiten und ihm diverse Verwaltungsaufgaben abnehmen. Beispielsweise lässt die Software nur regelkonforme Aktionen zu. Außerdem nimmt es dem Spielleiter viele Dinge ab, die sich nun die Software \glqq merkt\grqq{}. An gebotener Stelle wird diese Information dann dem Spielleiter angezeigt. Beispielhaft kann hier das Liebespaar genannt werden. Da der zweite Partner automatisch stirbt, wenn der erste stirbt, muss sich der Spielleiter das Liebespaar nicht mehr merken, da die Software das übernimmt. Sie benachrichtigt auch den Spielleiter, wenn der zweite Partner aus Liebeskummer stirbt. Als weiteres Beispiel ist zu nennen, dass die Software alle Opfer der Charaktere verwaltet. 

\medskip
Als Vorgabe dienen diesem Projekt die Spielanleitungen von \glqq Die Werwölfe von Düsterwald - Der Pakt\grqq{} und \glqq Die Werwölfe von Düsterwald - Die Gemeinde\grqq{}. Dadurch sind beispielsweise die Charaktere und Berufe, der Spielablauf, sowie die Spielregeln ähnlich wie bei einem Lastenheft vorgegeben. 

