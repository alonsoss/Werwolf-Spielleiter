\chapter{Reflexion}

\section{Reflexion Henrik Möhlmann}

Zu Beginn kannten wir uns teilweise noch nicht, daher war ich nach den Erfahrungen aus dem letzten Semester erst einmal skeptisch, allerdings hat sich sehr schnell ein gutes Arbeitsklima gebildet. Durch eine sehr gute Kommunikationsbereitschaft bei den Teammitgliedern konnten Probleme schnell identifiziert und Entscheidungen zügig getroffen werden. Ein wenig gedauert hat die Wahl des Projekts, aber auch da haben wir eine Lösung mit einem spannenden Projekt gefunden. 

Die gemeinsamen Treffen, ob es die offiziellen Praktikumstermine waren, oder die Planning, Standup, oder Dev-Treffen sind sehr harmonisch und konstruktiv verlaufen. Die Anwesenheitsquote war selbst bei kurzfristig nötigen Treffen nahezu bei 100\%. Das einzige, das etwas schwierig war, waren Treffen, wie die Retrospektive, da dort die Rollentrennung nicht hundertprozentig mit der Realität übereinstimmt, da wir ja quasi Produkt Owner, Kunde, Scrum Master, usw. alle auch als Teammitglieder haben und diese Funktionen auch rotiert haben. Daher war das nicht immer ganz einfach zu trennen, denn teilweise haben wir diverse Punkte bereits in internen Meetings angesprochen und behandelt. Was auch manchmal nicht einfach war, ist dass wir nicht Vollzeit mit Scrum an dem Projekt arbeiten konnten, da es ja nur eines von vielen war. Daher waren die Arbeitszyklen manchmal unterschiedlich verteilt, sodass von einem zum nächsten Standup Meeting bei einem Teammitglied sehr viel passiert ist, bei einem anderen weniger, beim nächsten Mal dann andersrum.  Daher ist das Pensum dann manchmal von einem zum nächsten Standup Meeting sehr unterschiedlich gewesen. 

Zu Beginn sind wir mit der Schätzung nicht ganz hingekommen, was zu einem großen Teil daran lag, dass wir den Aufwand für den Einstieg in JavaFX ein wenig unterschätzt haben. Da dort die Fehlermeldungen schlechter lesbar sind, als bei \glqq normalem Java\glqq{} hat dort die Fehlersuche einfach deutlich länger gedauert als gewohnt. Mit der Zeit haben wir JavaFX aber immer besser kennengelernt und ich persönlich habe auch nebenbei ein privates Projekt auf JavaFX umgestellt, da es gerade für größere Anwendungen letztlich doch angenehmer ist, als beispielsweise Java Swing. 

Besonders gut fand ich die Idee, relativ von Beginn an die Konzepte Charakter und Rolle zu trennen, das hat für eine wesentlich bessere Erweiterbarkeit mit zunehmendem Verlauf geführt, als wenn wir das nicht getrennt hätten. Auch sonst hat mir Scrum an sich sehr gut gefallen, da man in jedem Sprint Entscheidungen, die sich im Nachhinein als nicht perfekt herausgestellt haben noch wieder anpassen kann. Bei Programmieren nach dem Wasserfallmodell ginge das nicht so einfach, dann müsste man (mehr oder weniger) wieder von vorne anfangen. Besonders der Fakt, dass quasi die ganze Zeit ein funktionierender, sich im Funktionsumfang kontinuierlich vergrößernder, Prototyp vorhanden ist hat mir gut gefallen. So hat man immer etwas, das man spontan vorführen kann. 

Abschließend lässt sich sagen, dass das Programmieren nach Scrum dieses Semester eine sehr gute Erfahrung war. Es war ein sehr spannendes Projekt. Wir haben als Team sehr viel geschafft, sodass das Programm einen gewissen Umfang erreicht und ein ansprechendes Design hat. Durch die teilweise unterschiedlich tiefgehenden Kenntnisse der Teammitglieder in verschiedenen Bereichen hat sich das sehr gut ergänzt und man konnte einiges auch voneinander lernen. Ich persönlich habe beispielsweise einiges über Grafik und CSS gelernt, sowie im Verlaufe des Schreibens dieser Dokumentation auch einiges mehr über LaTeX. 

\section{Reflexion Janik Dohrmann}

Die Projektarbeit begann im ersten Sprint etwas chaotisch, da wir in diesem Sprint nicht wirklich gleichzeitig, sondern eher nacheinander bearbeiten mussten. Nachdem wir den ersten Sprint abgeschlossen hatten, war dieses Problem nicht mehr vorhanden, da wir die Grundstrukturen des Projekts erstellt hatten. Die Entscheidung JavaFX zu verwenden war eine gute Entscheidung da das Erstellen der View deutlich besser funktioniert hat als mit JavaSwing. In den weiteren Sprints haben wir viele Issues implementiert. Dadurch ist das Programm recht umfangreich geworden.

Einer der Punkte der meiner Meinung nach am meisten Probleme gemacht hat war wie schnell der Code unübersichtlich geworden ist, wenn man mit 6 Leuten an den gleichen Klassen gearbeitet hat. Dadurch wurde der Code auch schnell sehr lang und es hat etwas gedauert bis man gefunden hat was man gesucht hat.

Aus meiner Sicht war die Projektarbeit erfolgreich und konnte einen grundlegenden Eindruck in Scrum geben. Außerdem hat das Projekt einen guten Einblick in JavaFX und den Herausforderungen die dadurch entstehen gegeben. Die Gruppe hat gut zusammengearbeitet und die einzelnen Gruppenmitglieder hatten unterschiedliche Kenntnisse, die sich gut ergänzt haben. Durch das umfangreiche Programm konnten wir auch einen guten Eindruck in die Entwicklung eines Programms mithilfe der Scrum Methode bekommen.

\section{Reflexion Eric De Ron}

Die Gruppendynamik war ähnlich gut wie im vorherigen Softwaretechnik Projekt. Alle Mitglieder der Gruppe zeigten ein gleiches Interesse am positiven Ausgang des Projektes. Es wurde sich schnell für ein Projekt Idee entschlossen und von Anfang an arbeiteten alle gut mit. Der erste Sprint war ein bisschen holprig, dies hat sich jedoch in den folgenden Sprints verbessert, da die Zeit für einzelne Issues besser geschätzt wurde. Unsere geschätzte Zeit konnte meist eingehalten werden und zeigte uns, dass wir immer besser im Schätzen einzelner Aufgaben wurden.

Was auch gut war, dass wenn ein Mitglied ein Probleme hatte, ihm meist sofort geholfen wurde, so dass Issues auch meist im zeitlichen Rahmen geschlossen werden konnten. Natürlich wurde der Code bei so vielen Mitgliedern nach einer Zeit unübersichtlicher, dies konnte jedoch durch einige Refactoring-Prozesse teilweise gelöst werden.

In diesem Projekt habe ich den Grundlegenden Umgang von Scrum verstanden. Weitere Erfahrung konnte ich auch hinsichtlich JavaFX, Maven und CI sammeln, was für spätere Zwecke auf jeden Fall nützlich ist. Insgesamt hat die Zusammenarbeit Spaß gemacht und es konnten viele hilfreiche Erfahrungen gesammelt werden.

\section{Reflexion Florian Müller}

Die Ideenfindung für diese Projekt ging sehr schnell bei uns in der Gruppe, da wir uns schnell alle gut miteinander Verstanden haben. Wir hatten im ersten Sprint ein paar Probleme, da wir alle relativ zügig angefangen haben und dann manche Issues nicht komplett fertig gemacht werden konnte, da Teile davor nicht fertig waren, um Fehler zu finden die später aufgetreten sind.
Die Grundlagen vom Scrum habe ich in diesem Projekt verstanden und konnte dieses dort auch umsetzten. Im weiteren konnte ich meine Kenntnisse in Maven weiter vertiefen. In diesem Projekt hab ich komplett neue Erfahrungen in JavaFX sammeln können und auch umsetzen.


\section{Reflexion Matthias Hinrichs}
Insgesamt hat mir das Projekt sehr gut gefallen. Die Gruppendynamik war ziemlich gut und auch die Arbeitsteilung hat sehr gut funktioniert. Verschiedenste Probleme, die während des Projekts auftraten (z.B. mit JavaFX) wurden recht schnell durch die ganze Gruppe gelöst. Persönlich nutze ich für meine privaten oder studentischen Projekte mit Java im Normalfall als Build-Tool Gradle und hatte keine großen Erfahrungen mit Maven. Daher fand ich es auch mal ganz Interessant zu sehen, wie so ein Projekt mit Maven abläuft. In Zukunft bleibe ich allerdings nach Möglichkeit weiterhin bei Gradle. Auch mit JavaFX hatte ich bisher noch keine Erfahrungen gesammelt und war etwas überrascht, wie einfach es nach einer kurzen Einarbeitungszeit ist mit JavaFX seine Ideen umzusetzen. Seit Beginn des Softwaretechnikprojekts habe ich daher bereits ein kleineres Projekt mit JavaFX umgesetzt und ein weiteres größeres befindet sich gerade in der Entwicklung.

\section{Reflexion Alonso Essenwanger}
Wie bereits vorher von anderen Projektmitgliederen erwähnt, war auch ich zunächst skeptisch hinsichtlich einer effizienten Gruppenzusammenarbeit. Insbesondere Erfahrungen aus vorherigen Semestern und die aktuelle Situation (Covid-19), die unser Leben auf unterschiedliche Weise beeinflusst, ließen mich an einer funktionierenden Teamarbeit zweifeln. Diese Zweifel waren jedoch unbegründet und verflogen bereits nach der ersten Iteration. Die Online-Treffen klappten ohne große Umstände, die Aufgaben wurden fair und schnell aufgeteilt und bei Problemen waren die anderen Projektmitglieder stets zur Stelle. Unterstützt wurde dies durch die Scrum-Methode, die ich mir nun auch bei anderen Projekten gut vorstellen kann. Da ich bisher bei vielen Projekten mit Eclipse gearbeitet hatte, war meine erste Wahl auch bei diesem Projekt auf Eclipse gefallen. Allerdings kamen bereits bei der ersten Iteration Probleme auf. Meine Gruppenmitglieder haben mir daraufhin IntelliJ empfohlen.