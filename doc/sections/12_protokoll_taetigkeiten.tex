\chapter{Protokoll der Tätigkeiten der Teammitglieder}

\section{Tätigkeiten Eric De Ron}

\begin{tabularx}{\textwidth}{|l|p{4.5cm}|l|X|}
	\hline                                             % Gitterlinie oberhalb
	\textbf{\#}  &    \textbf{Bezeichnung}  &    \textbf{Typ}  & \textbf{Beschreibung}	 \\ 
	\hline \hline 
	\endhead      
	
	5 \label{iss:5}	&	Spieleranzahl soll eingegeben werden können &	Feature	&	Benutzer kann über einen Slider die Spieleranzahl anpassen  \\ \hline
	6 \label{iss:6}	&	Spielernamen sollen eingegeben werden können    &	Feature	&	Benutzer kann über Textfelder Namen der Mitspieler eingeben  \\ \hline
	8 \label{iss:8}	&	Die Opfer der Nacht sollen angezeigt werden &	Feature	&	Ein Element der View welches die Opfer der Nacht variabel anzeigt  \\ \hline
	18 \label{iss:18}	&	Gitignore erstellen &	Feature	&	Eine Git-Datei welche Intellij, Eclipse und NetBeansfiles Config-Dateien ausschließt  \\ \hline
	22 \label{iss:22}	&	Card select combobox crash  &	Bug	&	Die Comboboxen in der Card Select Szene wurden mit null initialisiert und erzeugten eine InvocationTargetException  \\ \hline
	23 \label{iss:23}	&	Konstanten in Konfigurationsdatei auslagern &	Feature	&	Benutzer kann Standardwerte der Software über eine JSON-Config-Datei anpassen, außerdem wurden alle Konstanten der Software ausgelagert  \\ \hline
	28 \label{iss:28}	&	Lynch Abstimmung Werwolf Auswahl bleibt bestehen	&	Bug	&	Ausgewählte Lynch Opfer waren auch in der Werwolf Opfer Anzeige ausgewählt  \\ \hline
	31 \label{iss:31}	&	Szenen Nummerierung / Bezeichnung in enum Klasse auslagern	&	Feature	&	Die Szenen Nummerierung des SceneManager wurden als Enum in eine separate Klasse ausgelagert  \\ \hline
	37 \label{iss:37}	&	Jäger einfügen	&	Feature	&	Implementieren des Jägers als Charakter mit eigender .fxml Datei. Hinzufügen eigender Szene in der Opfer Anzeige. Hinzufügen der Kurzanleitung.  \\ \hline
	46 \label{iss:46}	&	Zurück Button	&	Feature	&	In den Gamepreperation Szenen wird eine zurück Button angezeigt, welcher das zurück navigieren zu vorherigen Szenen erlaubt  \\ \hline
	59 \label{iss:59-2}	&	GUI Testfälle	&	Feature	&	GUI Testfälle ergänzt  \\ \hline
	69 \label{iss:69}	&	Wildes Kind einfügen    &	Feature	&	Implementieren des Wilden Kindes als Charakter mit eigender .fxml Datei. Hinzufügen eigender Szene in der ersten Nacht. Hinzufügen der Kurzanleitung.  \\ \hline
\caption{Tätigkeiten Eric De Ron}\label{tbl:eric}
\end{tabularx}

\newpage
\section{Tätigkeiten Janik Dohrmann}

\begin{tabularx}{\textwidth}{|l|p{4.5cm}|l|X|}
	\hline                                              % Gitterlinie oberhalb
	\textbf{\#}  &    \textbf{Bezeichnung}  &    \textbf{Typ}  & \textbf{Beschreibung}	 \\ 
	\hline \hline   
	\endhead
	
	2 \label{iss:2}	&	Dorfbewohner ins Spiel einfügen	&	Feature	&	Einfügen der Dorfbewohner ins Spiel  \\ \hline
	4 \label{iss:4}	&	Anzeigen Sieg einer Funktion	&	Feature	&	Implementieren der Anzeige für den Sieg einer Spielfunktion entsprechend der Regeln  \\ \hline
	10 \label{iss:10}	&	DoD erstellen	&	Feature	&	Erstellen der DoD in der Readme Datei im Repository  \\ \hline
	11 \label{iss:11}	&	Java Projekt erstellen	&	Feature	&	Erstellen des Java Projekts mit Java 13 und Maven  \\ \hline
	12 \label{iss:12}	&	CI Pipeline	&	Feature	&	Einrichten einer CI Pipeline für den Maven Test- und Buildprozess  \\ \hline
	30 \label{iss:30}	&	Package-Ordner Struktur ändern	&	Feature	&	Anpassen der Package/Order Struktur  an Aufgabenbereiche der Software\\ \hline
	39 \label{iss:39}	&	Seherin einfügen	&	Feature	&	Implementieren der Seherin als Charakter mit eigender .fxml Datei  \\ \hline
	50 \label{iss:50}	&	Maven package in CI Pipeline einfügen	&	Feature	&	In die CI Pipeline den Maven Befehl package einfügen  \\ \hline
	59 \label{iss:59-1}	&	GUI Testfälle	&	Feature	&	GUI Testfälle ergänzt  \\ \hline
	61 \label{iss:61}	&	Wenn alle Karten aufgedeckt sind wird durch Seherin ausgewählte Karte zugedeckt	&	Bug	&	Wenn alle Karten aufgedeckt sind wird durch Seherin ausgewählte Karte zugedeckt beim Einschlafen der Seherin zugedeckt \\ \hline
	63 \label{iss:63}	&	Vagabund einfügen	&	Feature	&	Implementieren des Vagabunden mit den Grundstrukturen für die Berufe und einer .fxml Datei für die Berufsauswahl   \\ \hline
	65 \label{iss:65}	&	Beichtvater einfügen	&	Feature	&	Implementieren des Beichtvaters mit einer .fxml Datei und einem Menüpunkt für die Tagaktionen der Berufe  \\ \hline
	71 \label{iss:71}	&	Anleitung zum Berufe programmieren schreiben	&	Feature	&	Eine Anleitung zum implementieren der Berufe im Wiki schreiben  \\ \hline
\caption{Tätigkeiten Janik Dohrmann}\label{tbl:janik}
\end{tabularx}

\newpage
\section{Tätigkeiten Alonso Essenwanger}

\begin{tabularx}{\textwidth}{|l|p{4.5cm}|l|X|}
	\hline                                              % Gitterlinie oberhalb
	\textbf{\#}  &    \textbf{Bezeichnung}  &    \textbf{Typ}  & \textbf{Beschreibung}	 \\ 
	\hline \hline 
	\endhead    
	
	1 \label{iss:1}	&	Werwölfe: Aufwachen	&	Feature	&	Implementiert die Option damit die Werwölfe aufwachen und sie farbig markiert werden  \\ \hline
	17 \label{iss:17}	&	Werwölfe: Einschlafen &	Feature	&	Implementiert die Option damit die Werwölfe einschlafen, die Markierung entfernt wird und es wird ein Text zum einschlafen der Werwölfe angezeigt  \\ \hline
	34 \label{iss:34}	&	Menüeintrag Spielanleitung/Infobox	&	Feature	&	Ein Menü Hilfe wurde erstellt mit der Spielanleitung und Info   \\ \hline
	49 \label{iss:49}	&	Farbiger Spielerstatus einführen 	&	Feature	&	Die Spieler besitzen ein farbiger Status. Die Karten werden zur Mitte des Spielfeldes angeordnet. Wenn ein Spieler stirbt wird die Karte grau angezeigt  \\ \hline
	55 \label{iss:55}	&	Hinweistext bei der Hexe	 &	Feature	&	Ein Hinweistext wird bei der Hexe angezeigt nur wenn sie noch töten darf  \\ \hline
	63 \label{iss:63-1}	&	Vagabund einfügen	&	Feature	&	Ein Menü für die Jobs wurde implementiert damit die Jobs angezeigt werden.  \\ \hline
	72 \label{iss:72}	&	Planung Usability Test	&	Feature	&	Usability Test mit LaTeX  \\ \hline
\caption{Tätigkeiten Alonso Essenwanger}\label{tbl:alonso}
\end{tabularx}

\newpage
\section{Tätigkeiten Matthias Hinrichs}

\begin{tabularx}{\textwidth}{|l|p{4.5cm}|l|X|}
	\hline                                              % Gitterlinie oberhalb
	\textbf{\#}  		& \textbf{Bezeichnung}				& \textbf{Typ}	& \textbf{Beschreibung}\\ \hline \hline  
	\endhead    	
	9 \label{iss:9}		& Spieler auf Spielfeld anzeigen	& Feature		& Für alle Spieler wird eine Karte auf dem Spielfeld angezeigt\\ \hline
	15 \label{iss:15}	& Karten umdrehen					& Feature		& Alle Spielerkarten können umgedreht werden\\ \hline
	29 \label{iss:29}	& CSS-Styles						& Feature		& Alle Stylings im Programm sind mit CSS umgesetzt\\ \hline
	32 \label{iss:32}	& FXML-Dateien umbauen				& Feature		& Alle FXML-Dateien sind so gebaut, dass sie möglichst kompakt sind, vertikal und horizontal zentriert werden und beim resizen des Programms nicht die Größe ändern\\ \hline
	33 \label{iss:33}	& Menüeintrag Beenden				& Feature		& Es gibt ein Menü mit einem Eintrag Beenden um das Programm zu beenden\\ \hline
	35 \label{iss:35}	& Menüeintrag Karten umdrehen		& Feature		& Der Menüeintrag um die Karten umzudrehen wird zu einem zusammengefasst\\ \hline
	41 \label{iss:41}	& Anzeigereihenfolge der Szenen		& Feature		& Das Boardlayout wird als erstes geladen und alle weiteren Szenen im Centerfield dargestellt\\ \hline
	51 \label{iss:51}	& Default Buttons					& Feature		& Alle weiterführenden Buttons werden als Default Buttons deklariert, damit sie mit Enter ausgelöst werden können\\ \hline
	52 \label{iss:52}	& Auflösungsoptimierung				& Feature		& Das Spiel ist für Full HD(1920x1080) und HD(1366x768) Bildschirme optimiert\\ \hline
	58 \label{iss:58}	& Markieren mit Rahmen				& Feature		& Spieler, die ausgewählt wurden, werden mit einem farbigen Rahmen markiert\\ \hline
	70 \label{iss:70}	& Fuchs einfügen					& Feature		& Es gibt den Charakter Fuchs entsprechend der Spielregeln\\ \hline
	73 \label{iss:73}	& Überlagernde Eingabefelder		& Bug			& Bei der Namenseingabe werden Eingabefelder überlagert\\ \hline
	74 \label{iss:74}	& Szenen zu groß					& Bug			& Bei verschiedenen Szenen kann es vorkommen, dass diese im Centerfield zu groß ist und den unteren Container der Spielerkarten aus dem Bild schiebt\\ \hline
\caption{Tätigkeiten Matthias Hinrichs}\label{tbl:matthias}
\end{tabularx}

\newpage
\section{Tätigkeiten Henrik Möhlmann}

\begin{tabularx}{\textwidth}{|l|p{4.5cm}|l|X|}
	\hline                                              % Gitterlinie oberhalb
	\textbf{\#}  &    \textbf{Bezeichnung}  &    \textbf{Typ}  & \textbf{Beschreibung}	 \\ 
	\hline \hline      
	\endhead
	
	7 \label{iss:7}	&	Kartenauswahl	&	Feature	&	Auswahl der Charakterkarten zu Spielbeginn  \\ \hline
	14 \label{iss:14}	&	Anzeige Lynch Opfer	&	Feature	&	Die Opfer der Lynch Abstimmung in der Mitte anzeigen und sterben lassen  \\ \hline
	19 \label{iss:19}	&	Spiel starten	&	Feature	&	Spiel initialisieren, Karten verteilen und mit erster Nacht beginnen; Backend initial programmieren  \\ \hline
	20 \label{iss:20}	&	Bilder Werwolf / Dorfbewohner	&	Feature	&	Bilder für Vorder- und Rückseite von Dorfbewohner und Werwolf einfügen  \\ \hline
	21 \label{iss:21}	&	Bug card select	&	Bug	&	Exceptions bei Klick auf \glqq weiter\grqq{} beheben  \\ \hline
	36 \label{iss:36}	&	Hexe einfügen (+ Anleitung Charakter Programmieren)	&	Feature	&	Hexe im Backend einfügen. Hexe im Frontend einfügen (inkl. eigener Scene). Anleitung zum Charakter programmieren schreiben.   \\ \hline
	38 \label{iss:38}	&	Amor einfügen	&	Feature	&	Amor und Liebespaar im Backend einfügen. Amor und Liebespaar im Frontend einfügen (jeweils inkl. eigener Scene).   \\ \hline
	40 \label{iss:40}	&	Kleines Mädchen einfügen	&	Feature	&	Kleines Mädchen im Backend einfügen. Kleines Mädchen im Frontend einfügen.   \\ \hline
	45 \label{iss:45}	&	Bug Spiel mit nur einer Fraktion	&	Bug	&	Wenn nur Charaktere einer Fraktion ausgewählt werden, das Spiel nicht starten, sondern gleich Gewonnen anzeigen.   \\ \hline
	47 \label{iss:47}	&	Highilight von Werwolfphase bleibt bei Lynch	&	Bug	&	Wenn die Werwölfe uneinig sind, die Markierungen zurücksetzen.   \\ \hline
	54 \label{iss:54}	&	Dieb einfügen	&	Feature	&	Dieb im Backend einfügen (inkl. Charakterwechsel). Dieb im Frontend einfügen (inkl. eigener Scene).   \\ \hline
	57 \label{iss:57}	&	Bug keiner kann gewinnen	&	Bug	&	Die Berechnung des Gewinns einer Fraktion anpassen, sodass der Gewinn zum richtigen Zeitpunkt angezeigt wird.   \\ \hline
	59 \label{iss:59}	&	GUI Testfälle	&	Feature	&	Initialen Satz GUI Testfälle erstellt  \\ \hline
	64 \label{iss:64}	&	Weißer Werwolf einfügen	&	Feature	&	Weißer Werwolf im Backend einfügen. Weißer Werwolf im Frontend einfügen (inkl. eigener Scene).   \\ \hline
	66 \label{iss:66}	&	Urwolf einfügen	&	Feature	&	Urwolf im Backend einfügen. Urwolf im Frontend einfügen (inkl. eigener Scene).   \\ \hline
	67 \label{iss:67}	&	Großer, böser Wolf einfügen	&	Feature	&	Großer, böser Wolf im Backend einfügen. Großer, böser Wolf im Frontend einfügen (inkl. eigener Scene).   \\ \hline
	68 \label{iss:68}	&	Wolfshund einfügen	&	Feature	&	Wolfshund im Backend einfügen. Wolfshund im Frontend einfügen. Einige GUI Tests durch JUnit Tests ersetzt.   \\ \hline
\caption{Tätigkeiten Henrik Möhlmann}\label{tbl:henrik}
\end{tabularx}

\newpage
\section{Tätigkeiten Florian Müller}

\begin{tabularx}{\textwidth}{|l|p{4.5cm}|l|X|}
	\hline                                              % Gitterlinie oberhalb
	\textbf{\#}  &    \textbf{Bezeichnung}  &    \textbf{Typ}  & \textbf{Beschreibung}	 \\ 
	\hline \hline      
	\endhead
	
	3 \label{iss:3}	&	Lynch Abstimmung: Spieler auswählen	&	Feature	&	Auswahl der Spieler für die Lynch Abstimmung ins Frontend und in das Backend (inkl. eigene Szene). \\ \hline
	13 \label{iss:13}	&	Lynch Abstimmung: Über Spieler abstimmen	&	Feature	&	Abstimmung über die Ausgewählten Spieler (eigene Szene) in Front- und Backend eingefügt.  \\ \hline
	16 \label{iss:16}	&	Werwölfe: Opfer suchen	&	Feature	&	Werölfe suche sich ihr Opfer aus. Teilt sich das ViewModel mit der Lynch Auswahl. (eigene Szene).  \\ \hline
	24 \label{iss:24}	&	Abstrakter Controller 	&	Feature	&	Abstraktes ViewModel angelegt um das anlegen von ViewModels zu vereinheitlichen. Alle Klassen angeschaut und dementsprechend angepasst.  \\ \hline
	42 \label{iss:42}	&	Refactoring	&	Feature	&	Klassen überarbeitet, vereinheitlicht    \\ \hline
	43 \label{iss:43}	&	Lynchchoosing: Tote Person kann zum lynchen ausgewählt werden	&	Bug	&	Anpassung der Auswahl \\ \hline
	44 \label{iss:44}	&	Werewolf choosing	&	Bug	& Überprüfung hinzugefügt ob Spieler tot sind  \\ \hline
	48 \label{iss:48}	&	Controller in ViewModel umbenennen	&	Feature	&	Umbenennung der bisherigen Klassen, welche explizite Logik beinhalten.  \\ \hline
	53 \label{iss:53}	&	Hauptmann einfügen	&	Feature	&	Hauptmann in das Front- und Backend hinzugefügt (inkl. Szene) und Lynch Abstimmung angepasst.   \\ \hline
	56 \label{iss:56}	&	Mehrere Runden hintereinander spielen	&	Feature	&	Runden Loop erstellt ohne, dass das Programm komplett geschlossen werden muss.  \\ \hline
	59 \label{iss:59-3}	&	GUI Testfälle	&	Feature	&	GUI Testfälle ergänzt  \\ \hline
	60 \label{iss:60}	&	Bug Alle Tot und einer Hauptmann	&	Bug	&	Bug fix, durch eine Abfrage wie viele noch leben.  \\ \hline
	62 \label{iss:62}	&	Bauer einfügen	&	Feature	&	Hauptmannwahl, TradeSelection angepasst und Bauer eingefügt.  \\ \hline
	75 \label{iss:75}	&	Bug: Deadlock Hauptmann mit Berufen	&	Bug	&	Eine Abfrage hinzugefügt.  \\ \hline
	76 \label{iss:76}	&	BUG zu wenig Stimmen bei Lynch Abstimmung	&	Bug	&	Anpassung in der Lynch Abstimmung, da eine Stimme zu wenig vorhanden war.    \\ \hline
	77 \label{iss:77}	&	BUG Jäger tot -> weiter Button sichtbar	&	Bug	&	Anpassung der Klasse PlayerVictimViewModel, da dort eine abfrage anders gesetzt werden musste.   \\ \hline
\caption{Tätigkeiten Florian Müller}\label{tbl:florian}
\end{tabularx}