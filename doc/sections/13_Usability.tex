\chapter{Usability}
\section{Vorwort}
Auf der bewaldeten Hochebene des Hunsrücks liegt etwas abseits der Wege das kleine Dorf Düsterwald. Seit einiger Zeit geschehen dort unheimliche Dinge: Hungrige Werwölfe streifen zwischen den Häusern umher! Nacht für Nacht fressen die Werwölfe einen Dorfbewohner. Die unheimlichen Gestaltwandler bleiben unentdeckt und leben tagsüber inmitten der Dorfbevölkerung. Die Dorfbewohner müssen herausfinden, wer unter ihnen sich nachts in einen Werwolf verwandelt, bevor sie alle diesem uralten Schrecken zum Opfer fallen. Jeder Spieler erhält eine Charakterkarte, die vorgibt, welchen Charakter der Spieler während einer Spielrunde verkörpert. Dies ist die geheime Identität (kurz: Identität). \\ ~\\ Wir stellen und jetzt eine Spielrunde vor, die dieses Rollenspiel spielt. Die Spieler sitzen im Kreis und Sie sind der Spielleiter und führen die Mitspieler durch die Partie. Die vorliegende Software soll Sie dabei unterstützen.

\section{Ablauf}    
    
\subsection{Zeitplanung}    
Wir werden uns an folgenden Plan halten:  \\
\begin{center}
\begin{tabular}{|l|l|}
  \hline
  Begrüßung & 3min \\ 
  Einverständniserklärung & 2min \\ 
  Aufgabendurchlauf & 60 min \\
  \hline \hline
  Fragebogen & 10min\\
  Abschied & 2min\\
  \hline
\end{tabular} 
\end{center}
	\subsection{Ablaufskript}
	\begin{itemize}
	\item Einführung: Ziel und Sinn des Tests erklären.

	\item Einverständniserklärung bereitlegen und unterzeichnen

	\item Ablauf grob erklären, also die folgenden Punkte

	\item Für jedes Proband:
	\begin{itemize}
	\item Aufgabenplan bereitlegen

	\item Spiel starten

	\item Sequentiell Aufgaben abarbeiten

	\item Spiel wieder schließen
	\end{itemize}

	\item Das zuletzt erstellte Spiel wird nun von allen X Probanden getestet

	\item Fragebogen bereitlegen und ausfüllen

	\item Verabschiedung
	\end{itemize}  
	
	\subsection{Aufgabenstellung}
Versuchen Sie, die folgenden Aufgaben sequentiell zu erfüllen.

\begin{large}
\begin{itemize}
    	\item Spielanleitung anzeigen lassen
    	\item Spiel konfigurieren, erfüllen Sie folgende Kriterien;
    	\begin{itemize}
    		\item Wählen Sie 5 Spieler aus
    		\item Benennen Sie alle Spieler mit ausgedachten Namen
    		\item Wählen Sie passende Charaktere Ihrer Wahl aus
    		\item Wählen Sie ggf. passende Berufe Ihrer Wahl aus
    	\end{itemize}
    	\item Starten Sie das Spiel
 		\begin{itemize}
 			\item Zeige die Charaktere an
 			\item Ggf. zeige die Berufe an
 			\item Spielen Sie bis eine Fraktion gewonnen hat
 		\end{itemize}   	
 		\item Starten Sie ein neues Spiel
 		\begin{itemize}
 			\item Starten Sie ein Spiel mit 15 Spielern 
 			\item Jeder Charakter soll in der Auswahl vorkommen 
 			\item Spiele mit Hauptmann
 			\item Spielen Sie mit Berufen
 			\item Es müssen 6 Bauern vorkommen
 			\item Starten Sie das Spiel 
 			\item Spielen Sie bis eine Fraktion gewonnen hat
 		\end{itemize}
    	\end{itemize}
\end{large}

	\subsection{Beschreibung der Testmaterialien}
    Folgend werden die Fragebögen und die Einverständniserklärung eine eigene Seite erhalten hier zuerst die Checkliste:
    \begin{itemize}
  \item Benötigte Materialen
  \begin{todolist}
    \item mindestens 3 Stifte
    \item Laptop/Computer
    \begin{todolist}
      \item installiertes Spiel
    \end{todolist}
    \item Fragebögen
  \end{todolist}
\end{itemize}

\newpage

\section{Anhang}
\begin{large}
\subsection{Einverständniserklärung}
Proband:\\

%\begin{itemize}
Name: \hrulefill

\hspace*{0mm}\phantom{Approved: }

\hspace*{0mm}\phantom{Approved: }

Vorname: \hrulefill

\hspace*{0mm}\phantom{Approved: }

\hspace*{0mm}\phantom{Approved: }

\vspace{2cm}

Hiermit erkläre ich mich mit den folgenden Punkten einverstanden:\\
\begin{itemize}
\item Ich bestätige meine Teilnahme an der Studie. Die Teilnahme erfolgt freiwillig.

\item Ich hatte ausreichend Zeit, Fragen zu stellen und wurde ausreichend informiert.

\item Ich verstehe, dass die hier ergobenen Daten anonymisiert werden und nicht an Dritte weiter gegeben werden.

\item Ich weiß, dass ich jederzeit meine Einverständniserklärung, ohne Angabe
von Gründen, widerrufen kann, ohne dass dies für mich nachteilige Folgen
hat.
\end{itemize}

Ich erkläre, dass ich mit der im Rahmen der Studie erfolgenden Aufzeichnung
von Studiendaten und ihrer Verwendung in anonymisierter Form einverstanden
bin.

\vspace{3cm}

\noindent\begin{tabular}{ll}
\makebox[2.5in]{\hrulefill} & \makebox[2.5in]{\hrulefill}\\
Datum, Ort & Unterschrift\\
\end{tabular}
%\end{itemize}
\end{large}

\newpage

\subsection{Fragebogen}
Bewerten Sie bitte die folgenden Aussagen anhand des gegebenem Schemas. Kreuzen Sie an.
\begin{large}


\begin{tabular}{|c|c|c|c|c|}
\hline
& + + & + & - & - -\\
\hline
Ich fand die Bedienung des Spielleiters sehr einfach & $\circ$ & $\circ$ & $\circ$ & $\circ$ \\
\hline
Ich habe die Regeln des Spielleiters verstanden & $\circ$ & $\circ$ & $\circ$ & $\circ$ \\
\hline
Die Einleitung war bei der Bearbeitung hilfreich & $\circ$ & $\circ$ & $\circ$ & $\circ$ \\
\hline
Ich wusste sofort, was zu tun ist & $\circ$ & $\circ$ & $\circ$ & $\circ$ \\
\hline
Das Spiel bedarf wenig Optimierung & $\circ$ & $\circ$ & $\circ$ & $\circ$ \\
\hline
\end{tabular}\\.


\textbf{Was hat Ihnen besonders gefallen?}\\

\framebox(450,75){}

\textbf{Was ist besonders negativ aufgefallen/hat Sie gestört?}\\

\framebox(450,75){}

\textbf{Sonstige Anmerkungen}\\

\framebox(450,75){}

\begin{center}
\textbf{Vielen Dank für Ihre Teilnahme!}
\end{center}
\end{large}

\newpage

\subsection{Angaben zur Person}

\begin{itemize}


\item \colorbox{gray!40}{\textbf{Wie viel Zeit verbringen Sie am Tag am Computer?}}
\begin{todolist}
    \item < 1 Stunde
    \item 1 - 3 Stunden
    \item >3 Stunden
\end{todolist}
\item \colorbox{gray!40}{\textbf{Kennen Sie 'Die Werwölfe von Düsterwald'?\hrulefill}}
\begin{todolist}
    \item Nein, nie gehört.
    \item Gehört, aber nie gespielt
    \item Ja, ich kenne die Regeln und den Ablauf
  \end{todolist}
\item \colorbox{gray!40}{\textbf{Falls vorherige Anwort 'Ja' lautet, welche Erweiterung kennen Sie?}}\\
\vspace{.2cm}


\noindent\begin{tabular}{l}
\makebox[2.5in]{\hrulefill}\\
\end{tabular}

\item \colorbox{gray!40}{\textbf{Spielen Sie regelmäßig Kartenspiele?}}\\
\vspace{.2cm}


\noindent\begin{tabular}{l}
\makebox[2.5in]{\hrulefill}\\
\end{tabular}
\end{itemize}

\newpage

\section{Ergebnisse}

\subsection{Eigene Beobachtungen}

Beim Durchführen des Tests ist die einfache Handhabung der
Benutzeroberfläche aufgefallen. So ist die Spielanleitung deutlich zu
erkennen und kann direkt und problemlos erkannt bzw. Gefunden werden.
Schwieriger wurde es bei Charaktere anzeigen, die nicht so schnell
gefunden werden konnte. Außerdem müsste noch optimiert werden, wie der Nutzer
erkennt, wer ausgewählt werden darf. \\
Schaut man sich die einzelnen Charaktere an, blieb unklar, wie die Hexe
ihre Mitspieler töten kann. Mit dem Fuchs wird das Liebespaar markiert,
was nicht notwendig wäre und es fehlt die Information, dass man ohne den
Dieb nicht in der Standardeinstellung spielen kann. Außerdem verschieben
sich die Berufe nach einem Neustart der App und werden so fehlerhaft
dargestellt. Daraus folgt, dass die Comboxen für die Berufe nicht
resetet waren. Des weiteren müsste noch ein Check für die Javaversion
eingebaut werden oder die vorhandene Version mit JavaVM gebündelt werden.

\subsection{Fragenbogen}

Die Tester fanden die Bedienung des Spielleiters durchweg einfach bis
sehr einfach. Ähnlich verhielt es sich mit dem Verständnis der Regeln
des Spielleiters. Das Spiel wurde von den Probanden als leicht
verständlich eingestuft und die Meisten gaben an. Dass es wenig
Optimierung bedarf. \\
Positiv bewertet wurden der einfache, nachvollziehbare Aufbau im
Gesamtbild, sowie das Farbkonzept und das Design des Programms. Aufgrund
des simplen Aufbaus blieben im Spielablauf keine Fragen offen, die
Bedienung wurde als einfach verstanden und der gesamte Spielablauf als
flüssig gelobt. Die Kurzanleitung, sowie hilfreiche Markierungen und
detaillierte Infos bei der Spielanleitung tragen zu einem problemlosen
Spielverständnis und der Freude am Spiel bei. Auch das interaktive
Design (umdrehen der Karten und die Verwendung der Originalmotive) wurde
von den Probanden gelobt. \\
Im Gegenzug negativ aufgefallen ist, wie die Stimmen ausgewählt wurden
und die Tatsache, dass über das Maximum hinaus stimmen abgegeben werden
konnten. Auch beim Auswählen der Charaktere fehlte es an einer
Limitierung der Zahlen. \\
Bezogen auf das Oberflächendesign hätten sich die Tester eine hellere
Hintergrundfarbe, sowie größere Schriften bei der Spielvorbereitung
gewünscht. Im Bezug auf die Karten hätten sie sich gewünscht, dass man
nicht alle Karten im Kopf haben müsste und, dass sie besser hätten
erklärt werden können. Bemängelt wurde außerdem, dass das Spiel für
einen Anfänger wohl zu kompliziert sei und es am visuellen Feedback
während des Spielens fehlte. \\ 
Da aber alle Tester das Spiel bereits in Kartenspielversion kannten
hatten sie dank ihrer Vorkenntnisse keine Probleme, den Ablauf zu
verstehen und haben sogar angegeben, dass es ein sehr gelungenes Spiel
sein, welches mit einem Design überzeugt, bei dem bloß noch Feinheiten
angepasst werden müssten. \newpage
Als allgemeine Verbesserungsvorschläge gaben die Probanden an, dass sie
es schön fänden, wenn sich die Hintergrundfarbe dem Tag-/Nachtzykluses
des Spieles anpassen würde und die Namensformen sich ändern würden, wenn
sie gehighlightet werden. Auch den bösen Wolf bei der Auswahl der Wölfe
hätten sie gerne gehighlightet. Auch zu den Charakterkarten gab es
Anmerkungen, wie man das Spiel mit dem abändern kleiner Nuancen noch
offensichtlicher und visuell ansprechender gestallten könnte. So sollten
die Charakterkarten beim Spielstart einmalig angezeigt werden und das
„Karten umdrehen“ sollte durch „Charakterkarten umdrehen“ verdeutlicht
werden. Für einmalige Charaktere wurde vorgeschlagen Radiobuttons
anzulegen. \\
Für das Spiel allgemein wurde angemerkt, dass eine Legende der
Markierungsfarben hilfreich wäre für ein besseres Verständnis des
Spielablaufes. Auch ein Mausoverlay bei den Karteninfos würde dazu
beitragen. Für einen besseren Überblick innerhalb des Spielverlaufes
wurde noch dazu ein Zeitstrahl vorgeschlagen, von dem man den nächsten
Spielzug ablesen könnte.



